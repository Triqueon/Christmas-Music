\documentclass[11pt]{article}
\usepackage[a5paper, landscape, margin=1.2cm, top=1cm]{geometry} % Papierformat
\usepackage[german]{babel} % Silbentrennung
\usepackage{multicol} % Zweispaltiges Inhaltsverzeichnis
\usepackage{hyperref} % Verlinkung des Inhaltsverzeichnis
\usepackage{tocloft} % Keine Seitenzahlen im Inhaltsverzeichnis
\usepackage[utf8]{inputenc} % Encoding
\usepackage{parskip} % Leerzeilen nach Absätzen
\cftpagenumbersoff{section} % Keine Seitenzahlen im ToC, s.o.
\setlength{\columnseprule}{1pt} % Spaltentrenner im ToC

%Neue Seite vor jedem Lied
\AddToHook{cmd/section/before}{\clearpage} % Neue Seite vor jedem Lied

\title{Das (digitale) Mettefrühstücksheft}
\author{
  Heribert Karst
  \\[3ex]
  \small Digitalisiert von Pascal Stein
}

\begin{document}

\maketitle
\newpage
\begin{multicols}{2}
\tableofcontents
\end{multicols}
\pagestyle{empty}

\section*{}

Liebe Geschwister,

Das Mettefrühstück ist seit langem eine feste Einrichtung in unserer Familie.
Sie wird auch von den Jüngeren und "Kleinen" geschätzt.
Dieses Treffen trägt viel zum Zusammenhalt bei, vor allem auch bei der "3. Generation".
Ohne das Mettefrühstück und das Cousins-Cousinen-Treffen -- letzteres ohne ersteres nicht denkbar --
wäre diese Generation sicher schon etwas auseinandergelaufen.

Ein besonderer Punkt beim Mettefrühstück ist stets das Singen von Weihnachtsliedern.
Damit der Gesang künftig noch inbrünstiger klinge, schenken wir Euch dieses Heft, das auch ein wenig zurückschaut
und durch einen Anhang den Überblick über die Massen erleichertn soll.
Es ist noch Platz für weitere Eintragungen!
Wir wünschen Euch noch frohe Weihnachten und uns allen miteinander noch viele Mettefrühstückslieder!

Heribert, Monika, Philipp und Tami

\section*{}

Liebe Familie,

seit obigem Vorwort von Heribert ist nun schon wieder viel Wasser den Rhein hinabgeflossen.
Über 25 Jahre später weiß nicht nur die von ihm zitierte 3., 
sondern mittlerweile auch die 4. Generation das Mettefrühstück und die assoziierten Familientraditionen zu schätzen,
und die 5. hat auch schon ihre ersten Vertreter.
Die letzten 25 Jahre haben auch einiges an technologischen Neuerungen mit sich gebracht,
und im Zeitalter von Smartphones und Tablets ist es sicher hilfreich, auch eine digitale Variante des lang erprobten Liederbuchs zu haben.

In der Hoffnung, dass damit der erweiterte Karstsche Chor auch weitere 25 Jahre vielstimmig erklingen möge,
wünsche ich euch allen frohe Weihnachten!

Pascal, Weihnachten 2024

\section*{Vom Mettefrühstück zum Weihnachtsbrunch}

Seit dem 7. Jahrhundert gibt es an Weihnachten drei Gottesdienste:
Die "Missa in nocte (ad galli cantum)", die "Missa in aurora" und die "Missa in die".
Der erste ist der heute meist Christmette genannte Gottesdienst.
Die Bezeichnung "ad galli cantum" (auf den Gesang kam es also schon immer an") zeigt,
daß die Messe ursprünglich in der späten Nacht, bzw. am ganz frühen Morgen, noch in der Nacht, gefeiert wurde.
Der Ausdruck "Mette" ist eine Eindeutschung des Wortes Matutin, der ersten Zeit des Stundengebetes.
Bei den strengen Nüchternheitsgeboten der früheren Zeit ist es verständlich,
daß nach dem Gottesdienst Bedarf nach einer Mahlzeit bestand -- und so entsand das Mettefrühstück.
Unsere Eltern luden dazu die nächsten Verwandten -- Eltern, Geschwister -- zu sich ein.
Dieser Zusammenhang ist heute nicht mehr gegeben, da wir ja die Mette meist am späten Abend des 24. Dezember feiern.
Da bestünde eher Bedarf nach einem mitternächtlichen Souper!

Unsere Eltern haben die Zahlbacher Mettefrühstückstradition nach Koblenz mitgenommen.
Zu Gast waren dann außer der Familie die Zimmers oder vielleicht eine Oma auf Besuch.
Nach der Rückkehr nach Zahlbach ging die Tradition ungebrochen weiter.
Mittlerweile hatte sich die Familie ja sehr vergrößert und der Tisch wurde immer länger
(beim letzten Mettefrühstück vor dem Umzug nach Koblenz waren es mit uns Kindern ca. 20 Personen).

Nachdem auch Thomas mit Fmailie am Wildgraben wohnte, halfen Renate und er stets fleißig mit,
schon damals waren sie eigentlich die Mettefrühstückswirtsleute.
Heute sprengt der mettefrühstücksberechtigte Personenkreis jeden Rahmen.
Wäre nicht der zeitversetzte Teilnehmeraustausch und gäbe es nicht das Ausweichquartier auf der Treppe -- wohin mit den Massen?
Und zeitlich zieht es sich von 11 bis 17 Uhr.
All das schreckt Renate und Thomas nicht, den Geist des Hauses zu pflegen
und damit einen ganz großen Beitrag zum Familienzusammenhalt zu leisten.

Dafür vielen herzlichen Dank, für Vergangenheit und Zukunft!!


\section{Es kommt ein Schiff, geladen}
\lilypondfile{eskommteinschiff.ly}

\section{Es ist ein Ros entsprungen}
\lilypondfile{esisteinros.ly}

\section{Das Kaschubische Weihnachtslied}
\lilypondfile{daskaschubische.ly}

\section{Lippai, steh auf vom Schlaf}
\lilypondfile{lippaistehauf.ly}

\section{Hört, es singt und klingt mit Schalle}
\lilypondfile{hörtessingt.ly}

\section{Als ich bei meinen Schafen wacht}
\lilypondfile{alsichbeimeinenschafen.ly}

\section{Zu Bethlehem geboren}
\lilypondfile{zubethlehemgeboren.ly}

\section{Lobt Gott, ihr Christen, alle gleich}
\lilypondfile{lobtgottihrchristen.ly}

\section{Kommet, ihr Hirten}
\lilypondfile{kommetihrhirten.ly}

\section{Singen wir mit Fröhlichkeit}
\lilypondfile{singenwirmit.ly}

\section{In dulci jubilo}
\lilypondfile{indulcijubilo.ly}


\section{Auf, Christen, singt festliche Lieder}
\lilypondfile{aufchristen.ly}

\section{Menschen, die ihr wart verloren}
\lilypondfile{menschendieihrwart.ly}

\section{Tochter Zion}
\lilypondfile{tochterzion.ly}

\section{O du fröhliche}
\lilypondfile{odufröhliche.ly}

\section{Heiligste Nacht}
\lilypondfile{heiligstenacht.ly}

\section{Maria durch ein' Dornwald ging}
\lilypondfile{mariadurcheindornwald.ly}

\section{Nun freut euch ihr Christen}
\lilypondfile{nunfreuteuch.ly}

\section{Mit einem Stern führt Gottes Hand}
\lilypondfile{miteinemstern.ly}
\end{document}